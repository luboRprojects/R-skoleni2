\documentclass[10pt,xcolor={dvipsnames}]{beamer}
\usepackage[utf8]{inputenc}
\usepackage[czech]{babel}
\usepackage[T1]{fontenc}
\usepackage{amsmath}
\usepackage{amsfonts}
\usepackage{amssymb}
\usepackage{url}
\begin{document}
\title{\textsf{R} - inferenční statistika}
\author{Lubor Homolka}
\date{\today}
\begin{frame}
\titlepage
\end{frame}

%----------

\begin{frame}
\frametitle{Obsah školení}
\begin{large}
Teoretická část:\\
\begin{itemize}
\item[--] Domácí cvičení  \\[0.3cm]
\item[--] Replikace a pravděpodobnost \\[0.3cm]
\item[--] Schéma testování hypotéz: vědecká vs. statistická \\[0.3cm]
\item[--] Chyby v interpretaci výsledků \\[0.3cm]
\item[--] Meta-analýza \\[0.5cm]
\end{itemize}
Praktická část:\\
\begin{itemize}
\item[--] proporční test \texttt{prop.test} \\[0.3cm]
\item[--] t-test a ANOVA \\[0.3cm]
\item[--] meta-analýza
\end{itemize}
\end{large}
\end{frame}

\begin{frame}
\frametitle{Domácí cvičení - dotazy}\begin{large}
\begin{center}
\url{https://github.com/luboRprojects}\newline\bigskip
	
\hspace*{-0.6cm}R-skoleni2 \hspace*{0.5cm}$\rightarrow$ \hspace*{0.5cm}skript: \texttt{plyr}$\_$\texttt{ukazka.R}
\end{center}
\end{large}
\end{frame}

%----------

\begin{frame}[fragile]
\frametitle{Frekvencionalistická statistika}
\begin{large}
\[\text{Pravděpodobnost} = \frac{\text{Příznivá pozorování}}{\text{Všecha pozorování}}\]\bigskip

\textcolor{WildStrawberry}{Chyba $\alpha$ = $\text{P}\left(\text{zamítáme}\ H_0 | H_0\ =\text{pravdivá} \right)$\bigskip
\[\alpha = \frac{\text{Počet článků, které zamítnou}}{\text{Všechny články}}\]}

\textcolor{NavyBlue}{Chyba $\beta$ = $\text{P}\left(\text{nezamítáme}\ H_0 | H_0\ =\text{nepravdivá} \right)$\bigskip
\[\beta = \frac{\text{Počet článků, které nezamítnou}}{\text{Všechny články}}\]}

\end{large}
\end{frame}

%----------

\begin{frame}
\frametitle{Hypotézy}
\begin{large}
\begin{itemize}
\item[--] Popperova falzifikace,
\item[--] Epistemologický boj: Neymar-Pearson a Fisher \\
\textcolor{WildStrawberry}{\item[--] Vědecká hypotéza: Organizace se správně implementovaným BSC, dosahují vyšší finanční výkonnosti (měřeno ROE).}\\
\textcolor{NavyBlue}{\item[--] Nulová hypotéza: Efekt implementace BSC na ROE neexistuje: $H_0: \beta = 0$ } 
\textcolor{NavyBlue}{\item[--] Alternativní hypotéza: Efekt implementace BSC na ROE je }\textcolor{WildStrawberry}{kladný}\textcolor{NavyBlue}{: $H_1: \beta > 0$ } 
\end{itemize}
\end{large}
\end{frame}

%----------


\begin{frame}
\begin{large}

\begin{itemize}
\item[--] nezveřejnění p-val $>0.05$.
\item[--] statistická významnost vs. věcná významnost
\item[--] kontrola splnění předpokladů - alternativní přístupy k asymptotickému odhadu (Monte Carlo)
\item[--] nepřesná interpretace intervalů spolehlivosti
\end{itemize}
\end{large}
\end{frame}

%----------

\begin{frame}
\frametitle{Meta-analýza: Analýza síly testu}
\begin{Large}
\begin{itemize}
\item[--] Jak ambiciózní je naše analýza?
\item[--] Jaké výsledky můžeme očekávat?
\item[--] Jak velký má být vzorek?
\end{itemize}
\end{Large}
\end{frame}

%----------

\begin{frame}
\frametitle{Analýza síly testu -- Power analysis}
\begin{large}
Analyzovaný problém souvisí s:
\begin{itemize}
\item[--] skutečným rozdílem mezi charakteristikami (effect size)
\item[--] neurčitost v chováním proměnných (variance)
\item[--] váha našeho důkazu (observations)
\item[--] požadovaná míra chybovosti ($\alpha$, $1-\beta$).
\end{itemize}
\end{large}
\end{frame}

%----------

\begin{frame}
\frametitle{Analýza síly testu -- Power analysis}
\begin{large}
Domníváme se, že průměrná hodnota ROE firem, které správně implementují BSC je \textcolor{NavyBlue}{o 3 \%} vyšší než u ostatních firem. Očekáváte, že \textcolor{NavyBlue}{rozptyl ROE je 36\%}. Protože nevíte, zda opravdu BSC přináší pozitivní efekt, musíte specifikovat toleranci k oběma chybám. Je nepřijatelné, abyste potvrdili, že vliv existuje v případě, kdy \textcolor{WildStrawberry}{neexistuje ($\alpha=0.05$)}. Naopak, až tolik nevadí, když nenajdete důkaz pro existenci efektu, \textcolor{WildStrawberry}{pokud existuje ($\beta = 0.2$), neboli síla testu je 0.8}. \newline\smallskip

\centering{Jaký je minimální počet respondentů pro náš výzkum?}
\end{large}

\end{frame}

%----------

\begin{frame}[fragile]
\frametitle{Analýza síly testu -- Power analysis}
Test rozdílu středních hodnot pro dvě úrovně je studentův t-test (\texttt{t.test}).\newline\smallskip

effect size pro t-test je stanoven jako: \[d=\frac{\mu_1 - \mu_2}{\sigma}\]

\begin{verbatim}
library(pwr) #install.packages("pwr")
effect.size <- (0.12-0.10)/0.06
pwr.t.test(d = effect.size , sig.level = 0.05, 
	      power = 0.8, type = c("two.sample"),
		alternative="greater" )
\end{verbatim}

Existuje velmi mnoho možností úprav, jako rozdílné velikosti souborů, rozdílné rozptyly apod.
\end{frame}

%----------

\begin{frame}[fragile]
\frametitle{Analýza síly testu -- Power analysis}
Výsledky:
\begin{verbatim}
     Two-sample t test power calculation 

              n = 111.9686
              d = 0.3333333
      sig.level = 0.05
          power = 0.8
    alternative = greater

NOTE: n is number in *each* group


\end{verbatim}

\textcolor{WildStrawberry}{Náš vzorek by musel být více než 220 firem, abychom dokázali detekovat rozdíl 2 \%!}

\end{frame}

%----------

\begin{frame}[fragile]
\frametitle{Analýza síly testu -- Váš úkol}
\begin{Large}

Neprovedli jste power analýzu a nyní Vám nic nevychází! Nasbírali jste totiž jen 80 firem v každé skupině (BSC, non-BSC). Jak velký by musel být skutečný efekt na ROE (rozdíl mezi $\mu_{\text{BSC}}$ a $\mu_{\text{non-BSC}}$), abyste tento rozdíl byli schopni nalézt?
\end{Large}\newline\smallskip

\url{github.com/luboRprojects/r_skoleni2/power_analysis.R}
\end{frame}

%----------

\begin{frame}[fragile]
\frametitle{Analýza síly testu -- Řešení}
\begin{verbatim}
> pwr.t.test(n = 80, sig.level = 0.05,  power = 0.8, 
    type = c("two.sample"), alternative="greater" )

     Two-sample t test power calculation 

              n = 80
              d = 0.3948399
      sig.level = 0.05
          power = 0.8
    alternative = greater

\end{verbatim}
\[0.3948399 = \frac{\text{efekt}}{0.06} \]
\textcolor{NavyBlue}{Efekt by musel být minimálně 2.369 \%, abychom ho dokázali detekovat.}
\end{frame}

%----------

\begin{frame}
\begin{figure}
\includegraphics[scale=0.3]{github}
\end{figure}
\bigskip
\url{github.com/luboRprojects/r_skoleni2/elementary_methods.R}
\vspace{1cm}
\end{frame}

%----------

\begin{frame}[fragile]
\frametitle{Proporční test}
Testuje hodnotu nebo shodu podílů $\pi$. Nulová hypotéza tvrdí, že neexistuje rozdíl $\text{H}_0: \pi_1 = \pi_2$\newline\smallskip

Zeptali jsme se 100 Čechů a 100 Slováků na preferenci daného výrobku.
\begin{itemize}
\item[--] Možné odpovědi: líbí/nelíbí
\item[--] 75\% Čechů se líbí, Slovákům pouze 70 \%
\item[--] Je možné tvrdit, že existují různé preference?
\end{itemize}
\hrule
\begin{verbatim}
> prop.test(x=c(75, 70), n=c(100, 100))
\end{verbatim}
\hrule\smallskip

Jak by se situace změnila, kdyby náš vzorek byl $10\times$ větší? (a proporce se zachovaly)

\end{frame}

%----------

\begin{frame}
\frametitle{Analýza kontingenční tabulky}
Zjišťujeme, zda struktura kontingenční tabulky získané z našich dat odpovídá té, kterou bychom očekávali nastala, kdyby neexistoval vztah mezi proměnnými.
\[CH = \sum\sum \frac{\left( n_{ij} - n_{ij}^*\right)^2}{ n_{ij}^*}\]


\begin{itemize}
\item[--] V případě splnění podmínek má testové kritérium $CH$ rozdělení $\chi^2$ 
\item[--] Porušení podmínek (min. počet pozorování v buňce 5) neumožňuje srovnání CH s $\chi^2$.
\item[--] V případě porušení podmínek buď \texttt{fisher.test}, nebo \texttt{parametr simulate.p.value=TRUE} ve funkci \texttt{chisq.test}.
\end{itemize}
\end{frame}

%----------

\begin{frame}[fragile]
\frametitle{Kontingenční tabulka - příklad}
Je prokazatelné, že se BSC firmy více hlásí k CSR?
\begin{verbatim}
> data.chi <- as.data.frame(matrix(c(30, 90, 60, 40),
	byrow=TRUE, ncol=2))
> data.chi 
        CSR-ne CSR-ano
BSC         30      90
non-BSC     60      40

> chisq.test(data.chi)
Pearson's Chi-squared test with Yates'
   continuity correction

data:  data.chi
X-squared = 26.2121, df = 1, p-value = 3.059e-07

\end{verbatim}
\end{frame}

%----------

\begin{frame}[fragile]
\frametitle{Korelační analýza}
Míra intenzity lineárního vztahu. Nulová hypotéza tvrdí, že vztah neexistuje, neboli $H_0: \rho=0$.
\begin{itemize}
\item[--] funkce \texttt{cor} vytváří korelační matici, ale netestuje významnost
\item[--] funkce \texttt{cor.test} provádí párové testy a přináší další možnosti, jako je např. volba typu korelačního koeficientu (Pearson, Spearman). 
\end{itemize}
\begin{verbatim}
cor(data.in[ ,c("Fresh", "Milk", "Grocery")])
cor.test(data.in$Fresh, data.in$Milk)
\end{verbatim}
\end{frame}

%----------

\begin{frame}[fragile]
\frametitle{Regresní analýza}
Analyzuje kauzální vztah mezi vysvětlovanou proměnnou a vysvětlujícími proměnnými. \newline\smallskip
Testujeme významnost jak celkového modelu, tak jednotlivých koeficientů.
\begin{itemize}
\item[--] hodnoty testů získáme ze \texttt{summary(objekt)}
\item[--] celkový F-test - je model lepší než průměr? $H_0: \beta_1=\ldots=\beta_k = 0$
\item[--] přispívá koeficient $\beta$ k vysvětlení? $H_0:\beta_l = 0$
\end{itemize}

\end{frame}

%----------

\begin{frame}
Prezentace a její obsah je samozřejmě reproducible!
\begin{figure}
\centering
\includegraphics[scale=0.7]{github_logo}
\end{figure}
\url{https://github.com/luboRprojects/r_skoleni2/}
\end{frame}



\end{document}